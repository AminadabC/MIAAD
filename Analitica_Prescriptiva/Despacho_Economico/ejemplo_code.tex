\documentclass[11pt]{article}
\usepackage[utf8]{inputenc} % Codificación UTF-8
\usepackage[T1]{fontenc}    % Codificación de fuentes
\usepackage[spanish]{babel} % Idioma español
\usepackage{xcolor}         % Colores
\usepackage{listings}       % Para mostrar código fuente

% ===== Configuración de estilos para Python =====
\definecolor{codegreen}{rgb}{0,0.6,0}
\definecolor{codegray}{rgb}{0.5,0.5,0.5}
\definecolor{codepurple}{rgb}{0.58,0,0.82}
\definecolor{backcolour}{rgb}{0.95,0.95,0.92}

\lstdefinestyle{mypython}{
    language=Python,
    backgroundcolor=\color{backcolour},
    commentstyle=\color{codegreen},
    keywordstyle=\color{magenta},
    numberstyle=\tiny\color{codegray},
    stringstyle=\color{codepurple},
    basicstyle=\ttfamily\footnotesize,
    breaklines=true,
    frame=single,
    captionpos=b,
    numbers=left,
    numbersep=5pt,
    showspaces=false,
    showstringspaces=false,
    showtabs=false,
}

\title{Modelo de Despacho Económico con Pyomo}
\author{Autor}
\date{\today}

\begin{document}

\maketitle

\section{Introducción}
Este documento presenta un modelo de despacho económico implementado en Python utilizando la biblioteca \texttt{Pyomo}. El código completo se muestra a continuación.

\section{Código del Modelo}
\begin{lstlisting}[style=mypython, caption={Modelo de Despacho Económico con Pyomo}]
from pyomo.environ import *

# Crear el modelo
model = ConcreteModel()

# Conjunto de generadores
model.Generadores = Set(initialize=['G1', 'G2', 'G3', 'G4', 'G5', 'G6'])

# Parámetros de los generadores
datos_generadores = {
    'G1': {'Pmax': 500, 'Pmin': 100, 'ai': 0.007, 'bi': 7, 'ci': 240, 'UR': 80, 'DR': 120, 'Pi': 440},
    'G2': {'Pmax': 200, 'Pmin': 50, 'ai': 0.0095, 'bi': 10, 'ci': 200, 'UR': 50, 'DR': 90, 'Pi': 170},
    'G3': {'Pmax': 300, 'Pmin': 80, 'ai': 0.009, 'bi': 8.5, 'ci': 220, 'UR': 65, 'DR': 100, 'Pi': 200},
    'G4': {'Pmax': 150, 'Pmin': 50, 'ai': 0.009, 'bi': 11, 'ci': 200, 'UR': 50, 'DR': 90, 'Pi': 150},
    'G5': {'Pmax': 200, 'Pmin': 50, 'ai': 0.008, 'bi': 10.5, 'ci': 220, 'UR': 50, 'DR': 90, 'Pi': 190},
    'G6': {'Pmax': 120, 'Pmin': 50, 'ai': 0.0075, 'bi': 12, 'ci': 190, 'UR': 50, 'DR': 90, 'Pi': 110}
}

# Definir parámetros en el modelo
model.Pmax = Param(model.Generadores, initialize={g: datos_generadores[g]['Pmax'] for g in model.Generadores})
model.Pmin = Param(model.Generadores, initialize={g: datos_generadores[g]['Pmin'] for g in model.Generadores})
model.ai = Param(model.Generadores, initialize={g: datos_generadores[g]['ai'] for g in model.Generadores})
model.bi = Param(model.Generadores, initialize={g: datos_generadores[g]['bi'] for g in model.Generadores})
model.ci = Param(model.Generadores, initialize={g: datos_generadores[g]['ci'] for g in model.Generadores})
model.UR = Param(model.Generadores, initialize={g: datos_generadores[g]['UR'] for g in model.Generadores})
model.DR = Param(model.Generadores, initialize={g: datos_generadores[g]['DR'] for g in model.Generadores})
model.Pi = Param(model.Generadores, initialize={g: datos_generadores[g]['Pi'] for g in model.Generadores})

# Parámetro de demanda total
demanda_total = 1263
model.Demanda = Param(initialize=demanda_total)

# Matriz B y coeficientes (Para el cálculo aproximado de las pérdidas de transmisión)
B1 = [[0.01 * val for val in row] for row in [
    [0.0017, 0.0012, 0.0007, -0.0001, -0.0005, -0.0002],
    [0.0012, 0.0014, 0.0009,  0.0001, -0.0006, -0.0001],
    [0.0007, 0.0009, 0.0031,  0.0000, -0.0010, -0.0006],
    [-0.0001, 0.0001, 0.0000,  0.0024, -0.0006, -0.0008],
    [-0.0005, -0.0006, -0.0010, -0.0006, 0.0129, -0.0002],
    [-0.0002, -0.0001, -0.0006, -0.0008, -0.0002, 0.0150]
]]

B2 = [0.001 * val for val in [-0.3908, -0.1297, 0.7047, 0.0591, 0.2161, -0.6635]]
B3 = 0.56

# Definir parámetros de pérdidas
model.B1 = Param(model.Generadores, model.Generadores, initialize={(g1, g2): B1[i][j] for i, g1 in enumerate(model.Generadores) for j, g2 in enumerate(model.Generadores)})
model.B2 = Param(model.Generadores, initialize={g: B2[i] for i, g in enumerate(model.Generadores)})
model.B3 = Param(initialize=B3)

# Variable de generación de cada generador
model.Pg = Var(model.Generadores, within=NonNegativeReals)

# Pérdidas
model.PL = Expression(expr=sum(model.B1[g1, g2] * model.Pg[g1] * model.Pg[g2] for g1 in model.Generadores for g2 in model.Generadores) +
                            sum(model.B2[g] * model.Pg[g] for g in model.Generadores) + model.B3)

# Restricciones de límites de generación
model.LimiteInferior = Constraint(model.Generadores, rule=lambda m, g: m.Pmin[g] <= m.Pg[g])
model.LimiteSuperior = Constraint(model.Generadores, rule=lambda m, g: m.Pg[g] <= m.Pmax[g])

# Restricciones de rampas
model.RampaSubida = Constraint(model.Generadores, rule=lambda m, g: m.Pg[g] - m.Pi[g] <= m.UR[g])
model.RampaBajada = Constraint(model.Generadores, rule=lambda m, g: m.Pi[g] - m.Pg[g] <= m.DR[g])

# Restricción de demanda considerando pérdidas
model.DemandaTotal = Constraint(rule=lambda m: sum(m.Pg[g] for g in m.Generadores) == m.Demanda + m.PL)

# Función objetivo: minimizar el costo de generación
model.CostoTotal = Objective(
    expr=sum(model.ai[g] * model.Pg[g] ** 2 + model.bi[g] * model.Pg[g] + model.ci[g] for g in model.Generadores),
    sense=minimize
)

# Resolver el modelo con IPOPT
solver = SolverFactory('ipopt')
result = solver.solve(model)

# Mostrar resultados
print("Resultados de la generación de cada unidad:")
for g in model.Generadores:
    print(f"Generación de {g}: {model.Pg[g].value:.2f} MW")

print(f"\nPérdidas totales: {model.PL():.2f} MW")
print(f"Costo total de generación: {model.CostoTotal.expr():.2f} $")
\end{lstlisting}

\end{document}