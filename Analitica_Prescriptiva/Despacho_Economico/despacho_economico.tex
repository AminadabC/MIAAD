\documentclass{article}
\usepackage{amsmath}

\begin{document}

\section{Descripcion del problema}
\subsection{Función Objetivo}

Al asignar razonablemente la producción de las unidades, los costos operativos pueden reducirse significativamente. 
El modelo de costos se simplifica a continuación:

\begin{equation}
    \min (F_T) =\min  \sum_{i=1}^{N} F_i(P_{Gi})
\end{equation}

Donde \( F_T \) indica el costo total del combustible, \( N \) representa el numero total de unidades de generación, 
\( P_{Gi} \) es la salida de potencia activa del \( i \)-th generador, y \( F_i(P_{Gi}) \) significa el costo del consumo 
de combustible correspondiente del \( i \)-th generador, y se puede expresar como un polinomio cuadrático dependiente de 
la potencian:

\begin{equation}
    F_i(P_{Gi}) = a_i(P_{Gi})^2 + b_i(P_{Gi}) + c_i
\end{equation}

donde \( a_i \), \( b_i \), and \( c_i \) representan los coeficientes de costo del combustible. Cuando la válvula de 
admisión de la turbomáquina se abre repentinamente, el resultado, conocido como 'efecto de punto de válvula', 
puede representarse generalmente como una función sinusoidal. Esta función sinusoidal se añadirá a la tradicional 
función de costo de combustible cuadrática, es decir:

\begin{equation}
    F_i(P_{Gi}) = a_i(P_{Gi})^2 + b_i(P_{Gi}) + c_i + |d_i \sin(e_i (P_{Gi}^{\min} - P_{Gi}))|
\end{equation}

Donde \( P_{Gi}^{\min} \) representa el límite inferior de la potencia de salida del \( i \)-th generador, 
y \( |d_i| \) y \( e_i \) son los coeficientes de costo. 

\section{Restricciones}
\subsection{Restricciones de Balance de Potencia}

% La potencia de salida de cada generador debe ser igual a la potencia de carga más la potencia de pérdida en 
% la línea de transmisión, es decir:

% \begin{equation}
    % P_{Gi} = P_{D} + P_{L}
% \end{equation}

Las restricciones de equilibrio de potencia son las restricciones más críticas en la operación de la unidad generadora. 
Si no se cumple esta restricción, se producirá la parálisis del sistema eléctrico y se amenazará seriamente la fiabilidad
 de la operación del sistema. Esta restricción se puede resumir en que la producción total de todas las unidades generadoras 
 debe ser igual a la suma de la carga y la pérdida de transmisión

\begin{equation}
    \sum_{i=1}^{N} P_{Gi} = P_{D} + P_{L}
\end{equation}

\begin{equation}
    P_{L} = \sum_{i=1}^{N}\sum_{j=1}^{N} P_{Gi}B_{ij}P_{Gj} + \sum_{i=1}^{N}B_{i0}P_{Gi} + B_{00}
\end{equation}

\subsection{Restricciones de Generación}

La potencia de salida de cada generador debe estar dentro de los límites de generación, es decir:

\begin{equation}
    P_{Gi}^{\min} \leq P_{Gi} \leq P_{Gi}^{\max}
\end{equation}

donde \( P_{Gi}^{\min} \) y \( P_{Gi}^{\max} \) son los límites inferior y superior de la potencia de salida del 
\( i \)-th generador, respectivamente.

\subsection{Límite de velocidad de rampa}
%La capacidad de rampa de cada generador debe estar dentro de los límites de rampa, es decir:

%\begin{equation}
    %\Delta P_{Gi}^{\min} \leq \Delta P_{Gi} \leq \Delta P_{Gi}^{\max}
%\end{equation}

Dentro de un período de tiempo, la variación de potencia está limitada por el funcionamiento del generador. 
Esta restricción se describe a continuación:

\begin{equation}
    - DR_{i} \leq  P_{Gi} - P_{Gi(t-1)} \leq UR_{i}
\end{equation}

Donde \( DR_{i} \) y \( UR_{i} \) son las tasas de disminución y aumento de potencia del \( i \)-th generador,
respectivamente.

\subsection{Límites de las zonas de operación prohibida}

La potencia de salida de cada generador debe estar fuera de las zonas de operación prohibida, es decir:

\begin{equation}
    P_{Gi}^{\min} \leq P_{Gi} \leq P_{Gi}^{\max}
\end{equation}

Los cojinetes del generador producirán vibraciones intensas cuando el generador funcione en algunas
zonas. Por lo tanto, la potencia de salida debe ajustarse para evitar zonas de funcionamiento prohibidas durante
la operación real. Los límites de las zonas de funcionamiento prohibidas se enumeran a continuación:

\begin{equation}
    \left\{
    \begin{array}{l}
        P_{Gi}^{\min} \leq P_{Gi} \leq P_{Gi}^{1} \\[8pt]
        P_{Gi}^{k-1} \leq P_{Gi} \leq P_{Gi}^{k}, \quad k = 2,3,\dots,n_Z \\[8pt]
        P_{Gi}^{n_Z} \leq P_{Gi} \leq P_{Gi}^{\max}
    \end{array}
    \right.
\end{equation}

\end{document}
