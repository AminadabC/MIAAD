\documentclass{beamer}
\usepackage[utf8]{inputenc}
\usepackage[T1]{fontenc}
\usepackage{amsmath}
\usetheme{Madrid}
\setbeamertemplate{footline}[frame number]  % Solo deja el número de página centrado abajo


% Información de la presentación
\title{Resolución de Sudoku mediante Optimización Lineal Entera}
\subtitle{Modelado matemático del problema de Sudoku}
\author{Aminadab Córdova Acosta} 
\institute{Universidad Autónoma de Ciudad Juárez \\
Maestría en Inteligencia Artificial y Analítica de Datos \\
Programación para Analítica Prescriptiva y de Apoyo a la Decisión \\
Instructor: Dr. Josué Domínguez Guerrero}
\date{\today}

\begin{document}

% Diapositiva de título
\begin{frame}
    \titlepage
\end{frame}

% Diapositiva de contenido
\begin{frame}{Contenido}
    \begin{itemize}
        \item Descripción del problema
        \item Modelo matemático
        \item Conclusiones
    \end{itemize}
\end{frame}

% Descripción del problema
\begin{frame}{Descripción del problema}
    \begin{itemize}
        \item Resolver un tablero de Sudoku 9x9 de forma automática mediante un modelo de optimización.
        \item Reglas del Sudoku:
        \begin{itemize}
            \item Cada fila debe contener los números del 1 al 9 sin repetir.
            \item Cada columna debe contener los números del 1 al 9 sin repetir.
            \item Cada subcuadro 3x3 debe contener los números del 1 al 9 sin repetir.
        \end{itemize}
        \item Se parte de un conjunto de pistas iniciales dadas.
        \item El objetivo es encontrar una solución válida que cumpla con todas las restricciones.
    \end{itemize}
\end{frame}

% Modelo matemático - Variables y objetivo
\begin{frame}{Modelo matemático: Variables y objetivo}
    \textbf{Variables:} 
    \[ x_{ijk} = \begin{cases} 1 & \text{si el número } k \text{ está en la celda } (i,j) \\ 0 & \text{en otro caso} \end{cases} \]

    \textbf{Función objetivo:} Minimizar el número de decisiones nuevas:
    \[ \min \sum_{(i,j,k)\ \text{no pistas}} x_{ijk} \]
\end{frame}

% Modelo matemático - Restricciones (parte 1)
\begin{frame}{Modelo matemático: Restricciones (1/2)}
    \begin{itemize}
        \item Una cifra por celda:
        \[ \sum_{k=1}^{9} x_{ijk} = 1 \quad \forall i,j \]
        \item Un número por fila:
        \[ \sum_{j=1}^{9} x_{ijk} = 1 \quad \forall i,k \]
        \item Un número por columna:
        \[ \sum_{i=1}^{9} x_{ijk} = 1 \quad \forall j,k \]
    \end{itemize}
\end{frame}

% Modelo matemático - Restricciones (parte 2)
\begin{frame}{Modelo matemático: Restricciones (2/2)}
    \begin{itemize}
        \item Un número por bloque 3x3:
        \[ \sum_{i=i_0}^{i_0+2} \sum_{j=j_0}^{j_0+2} x_{ijk} = 1 \quad \forall k,\ \text{bloques } (i_0,j_0) \]
        \item Pistas fijas:
        \[ x_{ijk} = 1 \quad \text{si la pista en } (i,j) \text{ es } k \]
    \end{itemize}
\end{frame}

% Conclusiones
\begin{frame}{Conclusiones}
    \begin{itemize}
        \item El Sudoku fue modelado como un problema de optimización entera binaria.
        \item Las restricciones del juego se expresaron como ecuaciones lineales.
        \item El modelo permite encontrar soluciones válidas desde un punto de vista matemático.
        \item Esta formulación es aplicable a problemas combinatorios similares de validación o asignación.
    \end{itemize}
\end{frame}

\end{document}
